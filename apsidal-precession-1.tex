\documentclass[usenatbib,twocolumn]{mnras}
%\linespread{2.0}
\usepackage[utf8]{inputenc}
\usepackage[T1]{fontenc}
\usepackage{graphicx}
\usepackage{longtable}
\usepackage{wrapfig}
\usepackage{rotating}
\usepackage[normalem]{ulem}
\usepackage{amsmath}
\usepackage{amssymb}
\usepackage{capt-of}
\usepackage{hyperref}
\usepackage{tensor}
\usepackage{amsmath}
\usepackage{caption}
\usepackage{tabularx}
\usepackage{subcaption}
\usepackage{pdfpages}
\usepackage{float}
\usepackage{booktabs}
\usepackage{enumitem}
\usepackage{graphicx}
\usepackage{tensor}
\usepackage{wasysym}
\usepackage{mathtools}
\usepackage{xcolor}
\usepackage{cancel}
\DeclarePairedDelimiter{\abs}{|}{|}
\DeclarePairedDelimiter{\norm}{||}{||}
\DeclarePairedDelimiter{\p}{(}{)}
\DeclarePairedDelimiter{\we}{\langle}{\rangle}

\title[Apsidal Precession in MMR]{Architecture Sculpting by Apsidal Precession in Mean Motion Resonances}
\author[Laune et al.]{ JT Laune,$^{1}$ Laetitia Rodet,$^{1}$ and Dong
  Lai$^{1}$
  \\
  $^{1}$Department of Astronomy, Cornell Center for Astrophysics and
  Planetary Science, Cornell University, Ithaca, NY 14853, USA \\}

\begin{document}
\maketitle
\section{Introduction}
    Planets have long been expected to migrate in their natal protoplanetary disks \citep[PPDs; e.g.][]{nelson00_migrat_growt_protop_protos_discs}.

\section{Coplanar Eccentric Restricted 3-Body Problem}
\label{sec:CER3BP}
The \emph{Coplanar Eccentric Restricted 3-Body Problem} (CER3BP) is the problem of a test particle orbiting a primary of mass $M$ near a $p:q$ MMR for a coplanar ($i=0^\circ$) perturber of mass $m_p$.

\subsection{Hamiltonian and Equations of Motion}
The non-dissipative problem for a $j+1:j$ resonance is a non-autonomous dimensionless Hamiltonian with two degrees of freedom,
\begin{align}
    \sqrt{a} &\longleftrightarrow \lambda\\
    \sqrt{a}(1-\sqrt{1-e^2})\simeq \frac12\sqrt{a}e^2&\longleftrightarrow -\varpi.
\end{align}
The Hamiltonian of the system is 
\begin{align}
    \label{eq:TPH}
    \mathcal H=-&\frac{GM}{2a}\nonumber\\
    +\frac{GMm_p}{a_p}&\left\{
        f_1(a,a_p)e\cos((j+1)n_p(a_p)t-j\lambda-\varpi)\right.\\
        &\left. +f_2(a,a_p)e_p\cos((j+1)n_p(a_p)t-j\lambda-\varpi_p)
    \right\}\nonumber
\end{align}

\subsection{Effects of apsidal precession}
\label{sec:}
We add the term $H_{\pi} = -\pi\Gamma\simeq = -\pi\left(\frac12\sqrt{\alpha}e^2\right)$ to the Hamiltonian \ref{eq:TPH}, which induces the precession term
\begin{equation}
    \left(\frac{d\varpi}{dt}\right)_{\rm ext} = \pi
\end{equation}

\section{Coplanar Comparable Mass Problem}
The \emph{Coplanar Comparable Mass Problem} (CCMP) is the problem of two coplanar comparable mass (mass ratio $q=m_1/m_2\sim\mathcal O(1)$) near a $p:q$ MMR.

\subsection{Hamiltonian and Equations of Motion}
The non-dissipative problem for a $j:j+1$ resonance is an autonomous dimensionless Hamiltonian with 4 degrees of freedom,
\begin{align}
    \sqrt{a_1}&\leftrightarrow \lambda_1\\
    \sqrt{a_1}(1-\sqrt{1-e_1^2})\simeq\frac12\sqrt{a_1}e_1^2&\leftrightarrow -\varpi_1\\
    \sqrt{a_2}&\leftrightarrow \lambda_2\\
    \sqrt{a_2}(1-\sqrt{1-e_2^2})\simeq\frac12\sqrt{a_2}e_2^2&\leftrightarrow -\varpi_2
\end{align}
The Hamiltonian of the system is
\begin{align}
    \mathcal H=
    &- \frac{G M m_{2}}{2 a_{2}} - \frac{G M m_{2} q}{2 a_{1}} \\
    &- \frac{G m_{2}^{2} q}{a_{2}} \left(e_{1} f_{1} \cos{\theta_1} + e_{2} f_{2} \cos{\theta_2}\right) \\
    &- \frac{G m_{2}^{2} q \left(e_{1} e_{2} f_{4} \cos{\left(g_{1} - g_{2} \right)} + f_{3} \left(e_{1}^{2} + e_{2}^{2}\right)\right)}{a_{2}},
\end{align}
where now there are two resonant arguments
\begin{align}
    \theta_1 &=\lambda_{2} \left(j + 1\right) - j \lambda_{1}-\varpi_{1}, \\
    \theta_2 &= \lambda_{2} \left(j + 1\right)- j \lambda_{1}-\varpi_{2}.
\end{align}





\clearpage
\bibliography{bibliography}
\bibliographystyle{mnras}
\clearpage
\onecolumn
\appendix
\section{Equations solved in the CER3BP}
We take the total energy of the particle as the Hamiltonian, as in Section \ref{sec:CER3BP} (equation \ref{eq:TPH}), and adopt the notation therein.
Utilizing a type 2 time-dependent generating function, $F=- \Theta_{0} \left(- j \lambda + \tau \left(j + 1\right)\right)$, we transform to the variables
\begin{equation}
    \Theta_{0} = \frac{\Lambda}{j} \longleftrightarrow
    \theta_0 = j \lambda - \tau \left(j + 1\right).
\end{equation}

The new Hamiltonian is
\begin{equation}
    \mathcal H_0 = - \Theta_{0} j - \Theta_{0} + e_{p} f_{2} \mu \cos{\left(\theta_{0} \right)} + \sqrt{2} f_{1} \mu \sqrt{\frac{\sqrt{X^{2} + Y^{2}}}{\Theta_{0} j}} \cos{\left(\theta_{0} - \operatorname{atan}_{2}{\left(Y,X \right)} \right)} - \frac{1}{2 \Theta_{0}^{2} j^{2}},
\end{equation}
and, by Hamilton's equations,
\begin{align}
   \dot\theta_0 &= - j - 1 - \frac{\sqrt{2} f_{1} \mu \sqrt{\frac{\sqrt{X^{2} + Y^{2}}}{\Theta_{0} j}} \cos{\left(\theta_{0} - \operatorname{atan}_{2}{\left(Y,X \right)} \right)}}{2 \Theta_{0}} + \frac{1}{\Theta_{0}^{3} j^{2}}\\
   \dot\Theta_0 &= \mu \left(e_{p} f_{2} \sin{\left(\theta_{0} \right)} + \sqrt{2} f_{1} \sqrt{\frac{\sqrt{X^{2} + Y^{2}}}{\Theta_{0} j}} \sin{\left(\theta_{0} - \operatorname{atan}_{2}{\left(Y,X \right)} \right)}\right)\\
   \dot X &= \frac{\sqrt{2} f_{1} \mu \sqrt{\frac{\sqrt{X^{2} + Y^{2}}}{\Theta_{0} j}} \left(2 X \sin{\left(\theta_{0} - \operatorname{atan}_{2}{\left(Y,X \right)} \right)} + Y \cos{\left(\theta_{0} - \operatorname{atan}_{2}{\left(Y,X \right)} \right)}\right)}{2 \left(X^{2} + Y^{2}\right)} \\
   \dot Y &= \frac{\sqrt{2} f_{1} \mu \sqrt{\frac{\sqrt{X^{2} + Y^{2}}}{\Theta_{0} j}} \left(- X \cos{\left(\theta_{0} - \operatorname{atan}_{2}{\left(Y,X \right)} \right)} + 2 Y \sin{\left(\theta_{0} - \operatorname{atan}_{2}{\left(Y,X \right)} \right)}\right)}{2 \left(X^{2} + Y^{2}\right)}
\end{align}

\section{Comparable Mass Problem}

\end{document}