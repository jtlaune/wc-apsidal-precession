\documentclass{article}
\usepackage[margin=1in]{geometry}
\usepackage{amsmath,amssymb}
\usepackage{graphicx}
\usepackage{natbib}
\bibliographystyle{unsrtnat}

\title{MMR Sub-resonances and Apsidal Precession}
\author{JT Laune}

\begin{document}
\maketitle

\section{Introduction}
An orbiter in a potential with a nonzero quadrupole moment will undergo apsidal precession. 
Common astrophysical scenarios with a nonzero quadrupole include a planet being perturbed secularly by a disk or another planet as well as a planet orbiting an oblate star (with a $J_2$ moment). 
In the case of a disk, a planet's longitude of perihelion $\varpi$ will precess at a rate
\begin{equation}\label{eq:diskprecscaling}
    \frac{\dot\varpi}{n} \simeq 0.0004\left(\frac{a}{1~\rm{au}}\right)^3\left(\frac{M_d/M}{0.1}\right)\left(\frac{3~\rm{au}}{r_{\rm in}}\right)^2\left(\frac{10~\rm{au}}{r_{\rm out}}\right)
%= \frac34 \frac{n}{M}\int_{r_{\rm in}}^{r_{\rm out}} \Sigma(r_d)\left(\frac{a}{r_d}\right)^3 dm_d
\end{equation}
for a star of mass $M$, a disk of mass $M_d$ extending from $r_{\rm in}$ to $r_{\rm out}$ ($\gg r_{\rm in}$), and planet at semimajor axis (SMA) $a$ and mean motion $n$.
Apsidal precession affects MMR angles of the form $\theta_i=(j+1)\lambda_2-j\lambda_2-\varpi_i$ by adding a precession term $\dot\varpi_i$.

Here we investigate the MMR capture of a test particle by a planet and then demonstrate the feasibility of precession-induced resonance overlap disrupting capture in the comparable mass problem.

\section{One Planet and a Test Particle}\label{sec:tp}
\subsection{Hamiltonian in Orbital Elements}
Consider a test particle outside of a planet of mass $m_p$.
The subscript $p$ denotes the planet $m_p$'s orbital elements.
Let the secular precession frequency of the test particle's longitude of perihelion be $\omega$ and the planet's $\omega_p$.
Suppose that $\varpi_p=0$ at $t=0$.
We'll work in angular time, $\tau = n_pt$, where $t$ is physical time.

The Hamiltonian in orbital elements is
\begin{equation}
    \label{eq:TPH}
    \mathcal H=-\frac{GM}{2a}
    -\frac{GMm_p}{a_p}\left(
    f_1e_p\cos(\theta_0 - \omega_p\tau)
    +f_2e\cos(\theta_0 - \varpi)
    \right)
    + \frac12\sqrt{a}e^2\omega.
\end{equation}
where
\begin{equation}
    \theta_0    =(j+1)\lambda -j\tau
\end{equation}
and $f_1$, $f_2$ are constants.

\subsection{Dissipation}
We add dissipative terms
\begin{align}
    \frac{\dot e}{e} & = - \frac{1}{T_e},                   \\
    \frac{\dot a}{a} & = -\frac{1}{T_m} - \frac{2e^2}{T_e}. \\
\end{align}

\subsection{Hamiltonian in Canonical Variables}
Scaled by $n_p=\sqrt{GM/a_p^3}$ and expressed in the canonical variables
\begin{align}
    \Theta=\frac{\sqrt{a/a_p}}{j+1}           & \longleftrightarrow \theta_0 \\
    \Gamma\simeq\frac12\sqrt\frac{a}{a_p} e^2 & \longleftrightarrow -\varpi, \\
\end{align}
the Hamiltonian is
\begin{equation}
    \mathcal H =
    - \frac{1}{2 \Theta^{2} \left(j + 1\right)^{2}}
    - \Theta j - \omega\Gamma
    - e_{p} f_{1} \mu_{p} \cos{\left(\theta_0-\omega_{p} \tau \right)}
    - f_{2} \mu_{p} \sqrt{\frac{2\Gamma}{(j + 1)\Theta}} \cos(\theta_0 -\varpi).
\end{equation}

%It's often useful to work in the Canonical cartesian coordinates, which we'll denote by
%\begin{equation}
%    X=\sqrt{2\Gamma}\sin(-\varpi) \longleftrightarrow Y=\sqrt{2\Gamma}\cos(-\varpi).
%\end{equation}

\subsection{The $\theta_0-\omega_p \tau$ resonance}
Let's first consider the $\theta_0-\omega_p\tau$ resonance in isolation, i.e. consider $f_2=0$.

Two canonical transformations bring us to the following Hamiltonian,
\begin{align}
    \mathcal H_1 = -\frac{1}{2(j+1)^2(\Theta_1-\Theta_c)^2} - j(\Theta_1-\Theta_c) - \omega_p(\Theta_1-\Theta_c) - e_pf_1\mu_p\cos\theta_1
\end{align}
with canonical variables
\begin{align}
    \Theta_1 = \Theta-\Theta_c \longleftrightarrow \theta_1=\theta_0-\omega_p\tau,
\end{align}
where $\Theta_c$ is the critical value of $\Theta$ such that
\begin{align}
    \dot\theta_1 & = \left(\frac{\partial\mathcal H_1}{\partial\Theta_1}\right|_{\Theta_1=0} = 0.
    %\Rightarrow \Theta_c & = \frac{1}{(j+1)^{2/3}(j+\omega_p)^{1/3}}
\end{align}
Near resonance, $\Theta_1/\Theta_c\ll 1$ and we can expand the Hamiltonian and drop constant terms,
\begin{align}
    \mathcal H_1 & \approx -\frac{1}{2(j+1)^2\Theta_c^2}\left(1-\frac{2\Theta_1}{\Theta_c}+\frac{3\Theta_1^2}{\Theta_c^2}\right)
    +j\Theta_c-j\Theta_1 +\omega_p\Theta_c - \omega_p\Theta_1 - e_pf_1\mu_p\cos\theta_1                                          \\
                 & \simeq -\frac{3\Theta_1^2}{2(j+1)^2\Theta_c^4} - e_pf_1\mu_p\cos\theta_1
\end{align}
The resonance width is the width of the separatrix through the points $(\Theta_1=0,\theta_1=\pm\pi/2)$,
\begin{equation}\label{eq:d1}
    \delta_1 = \frac{\delta n}{n_0} = \frac32\frac{\delta a}{a_0} = \frac{6}{(j+1)^{2/3}(j-\omega_p)^{1/3}}\sqrt{\frac23 e_p |f_1| \mu_p},
\end{equation}
where $a_0=((j+1)/(j-\omega_p))^{2/3}a_p$ and $n_0=a_0^{3/2}$.

\subsection{The $\theta_0-\varpi$ resonance}
Now we consider the $\theta_0-\varpi$ resonance in isolation, i.e. consider $f_1=0$.

A canonical transformation brings us to the Hamiltonian
\begin{equation}
    \mathcal H_2 =
    - \frac{3 \Gamma_{2}^{2}}{2 \Theta_{2}^{4} \left(j + 1\right)^{2}}
    + \Gamma_{2} \left(\frac{1}{\Theta_{2}^{3} \left(j + 1\right)^{2}}-j\right)
    - \frac{\sqrt{2} \sqrt{\Gamma_{2}} f_{2} \mu_{p} \cos{\left(\gamma_{2} \right)}}{\sqrt{j + 1}}
\end{equation}
with canonical variables
\begin{align}
    \Gamma_2 = \Gamma=\frac12\sqrt ae^2                                     & \longleftrightarrow \gamma_2 = \theta_0 -\varpi \\
    \Theta_2 = \Theta - \Gamma_2 = \frac{\sqrt a}{j+1} - \frac12\sqrt a e^2 & \longleftrightarrow \theta_0.
\end{align}
Since $\theta_0$ is not in the transformed Hamiltonian, $\Theta_2$ is a constant of motion.

The coefficient of the middle term is zero at exact resonance.
The scale of resonance is set by the first and third terms being equal,
\begin{equation}
    e_0^2 \sim\frac{1}{\sqrt{a_0}(j+1)^{5/3}} \left(\frac83 a_0^2|f_2|\mu_p\right)^{2/3}
\end{equation}
The conserved quantity, $\Theta_2$, sets the width,
\begin{equation}\label{eq:d2}
    \delta_2 = \frac{\delta n}{n_0} = \frac32\frac{\delta a}{a_0} \sim 3(j+1)e_0^2
    = \frac{3}{\sqrt{a_0}(j+1)^{2/3}}\left(\frac83 a_0^2 |f_2|\mu_p\right)^{2/3}
\end{equation}

\subsection{Dissipative capture for $\omega_p>0$}
We set $\omega=0$ and choose a value for $\omega_p$ and $e_p$.
To test capture, we set $T_e=10^{3} (2\pi/n_p)$ and $T_m=10^{6} (2\pi/n_p)$, start the test particle at $a=1.4 a_p$, and integrate for $T=10^6(2\pi/n_p)$.
We plot the capture outcome (black for capture, red for escape) in the range $\omega_p/n_p \in [-10^{-2},10^{-2}]$, $e_p\in [0.0,0.03]$ in Figure~\ref{fig:grid}.
The resonance widths $\delta_1$ and $\delta_2$ are plotted for reference.
Roughly, capture is interrupted for $e_p>0.01$ and $5\times10^{6}\lesssim|\omega_p|\lesssim\delta_1$

In Figure~\ref{fig:ex1}, we give examples of capture and non-capture for the eccentricity $e_p=0.02$ (plotted in yellow in Figure~\ref{fig:grid}).

\begin{figure}
    \centering
    \includegraphics[width=0.4\textwidth]{Figure 34.png}
    \caption{Capture outcomes for $\omega_p/n_p$ and $e_p$. Red squares indicate escape; black squares indicate capture. Yellow squares are the examples presented in Figure~\ref{fig:ex1}. The resonance widths $\delta_1$ (equation~\ref{eq:d1}) and $\delta_2$ (equation~\ref{eq:d2}) are plotted in magenta an cyan, respectively.}
    \label{fig:grid}
\end{figure}
\begin{figure*}
    \centering
    \includegraphics[width=0.3\textwidth]{Figure 50.png}
    \includegraphics[width=0.3\textwidth]{Figure 53.png}
    \includegraphics[width=0.3\textwidth]{Figure 57.png}
    \caption{
        \textit{Left:} Capture for $\omega_p/n_p=2.68\times10^{-6}$. The precession frequency is not large enough for resonance interference.
        \textit{Middle:} Escape for $\omega_p/n_p=1.39\times10^{-4}$. Eccentricity is excited near resonance but then the planet passes through the resonance.
        \textit{Right:} Capture for $\omega_p/n_p =3.73\times10^{-3}$. The precession frequency is large enough so that the resonances are well separated.
    }
    \label{fig:ex1}
\end{figure*}

\subsection{Reducing rotation}
The canonical transformation which combines the $e_p$ and $e$ terms in equation~(\ref{eq:TPH}) presented in \citet{wisdom_canonical_1986} \citep[c.f.][]{moutamid14_coupl_between_corot_lindb_reson,laune22_apsid_align_anti_align_planet} is also applicable in the apsidally precessing case.
Through a similar series of canonical transformations,
\begin{align}
    \mathcal H_3
     & =- \frac{3 \Theta_{0}^{2} j^{\frac{2}{3}} \left(j + 1\right)^{\frac{4}{3}}}{2}\nonumber                                                                                   \\
     & -\sqrt{2\Phi} \left(f_{2} \mu_{p} \cos{\left(\phi - \theta_{0} \right)}+\frac{ e_{p} f_{1} \left(\omega_p-\omega\right) \cos{\left(\phi \right)}}{f_{1}}\right) \nonumber \\
     & +\Theta_{0} \omega- \Phi \left(\omega_p - \omega\right)
\end{align}
where the canonically conjugate coordinate--momentum pairs are
\begin{align}
    \phi       & = \operatorname{acot}{\left(\frac{\cos{\left(\varpi-\omega_p \tau \right)}}{\sin{\left(\varpi-\omega_p \tau  \right)}} - \frac{e_{p} f_{1}}{\sqrt[4]{\alpha} e f_{2} \sin{\left(\varpi-\omega_p \tau  \right)}} \right)} \\
               & \longleftrightarrow\Phi = \frac{\sqrt{\alpha} e^{2}}{2} + \frac{\sqrt[4]{\alpha} e e_{p} f_{1} \cos{\left(\omega_p \tau - \varpi \right)}}{f_{2}}  + \frac{e_{p}^{2} f_{1}^{2}}{2 f_{2}^{2}}                             \\
    \theta_{0} & = (j+1) \lambda - j\tau                                                                                                                                                                                                  \\
               & \longleftrightarrow\Theta_{0} = \frac{\sqrt{\alpha}}{j+1} - \frac{1}{(j+1)^{\frac{2}{3}} \sqrt[3]{j}}.
\end{align}
We see that in the case $\omega_p=\omega$, the Hamiltonian reduces to a single critical argument just as in the non-precessing case (with a slight shift of the term $\omega\Theta_0$).

\section{Two Planets}
\subsection{Hamiltonian in Orbital Elements}
Denote the inner planet's orbital elements with a subscript of $1$ and the outer planet's orbital elements with a subscript of $2$.
Let $q=m_1/m_2$ be the mass ratio.
Let $\omega_1$ and $\omega_2$ be the apsidal precession rates of the two planets.
The Hamiltonian is
\begin{equation}
    \mathcal H=
    - \frac{G M m_{1}}{2 a_{1}}
    - \frac{G M m_{2}}{2 a_{2}}
    - \frac{G m_{1}m_2}{a_{2}} \left(e_{1} f_{1} \cos(\theta_0 - \varpi_1) + e_{2} f_{2} \cos(\theta_0-\varpi_2)\right)
    - \frac12\sqrt{\alpha_1}e_1^2\omega_1
    - \frac12\sqrt{\alpha_2}e_2^2\omega_2,
\end{equation}
where $\theta_0 = (j+1)\lambda_2-j\lambda$.

\subsection{Dissipation}
We add dissipation to the system with the terms
\begin{align}
    \frac{da_1}{dt} & =- \frac{a_{1}}{T_{m,1}} - \frac{2 a_{1} e_{1}^{2}}{T_{e,1}}, \\
    \frac{da_2}{dt} & =- \frac{a_{2}}{T_{m,2}} - \frac{2 a_{2} e_{2}^{2}}{T_{e,2}}, \\
    \frac{de_1}{dt} & =- \frac{e_{1}}{T_{e,1}},                                     \\
    \frac{de_2}{dt} & =- \frac{e_{2}}{T_{e,2}}.
\end{align}

\subsection{Dissipative capture and disruption}
We set $m_1=3\times10^{-6}$ and $m_2=6\times 10^{-6}$ ($q=1/2$).
We set $T_{m,1}=\infty$, $T_{m_2} = 10^{7}(2\pi/n_{1,0})$ and $T_{e,1} = 2\times10^4(2\pi/n_{1,0})$, $T_{e_2} = 10^4(2\pi/n_{1,0})$, where $n_{1,0}$ is the initial mean motion of planet 1.
We integrate for $T=10^6(2\pi/n_{1,0})$.

From the results of Section~\ref{sec:tp}, we expect resonance overlap and capture disruption around $\omega_2-\omega_1\sim\mu_1^{2/3}n_{1,0}$, i.e. near $\sim 2\times10^{-4}n_{1,0}$.
In Figure~\ref{fig:cm}, we demonstrate capture in the non-precessing case (left) and escape in the precessing case with $\omega_2-\omega_1=10^{-4}n_{1,0}$ (right).

\begin{figure}
    \centering
    \includegraphics[width=0.4\textwidth]{cm-control-new.png}
    \includegraphics[width=0.4\textwidth]{cm-miss.png}
    \caption{Mean motion resonance encounter for $m_1/M=3\times10^{-6}$ and $m_2/M=6\times10^{-6}$ ($q=1/2$). The two planets are initiated just wide of resonance at a period ratio of $P_2/P_1=1.55$.
    \textit{Left:} An example of capture in the non-precessing case, $\omega_1=\omega_2=0$. After around $2\times10^5(2\pi/n_{1,0})$, the planets cross the 2:3 MMR and are captured, indicated by the tight libration of $\theta_1$ and $\theta_2$. The periapses transition from being aligned ($\Delta\varpi=\varpi_1-\varpi_2=0^\circ$) to anti-aligned ($\Delta\varpi=\varpi_1-\varpi_2=180^\circ$).
    \textit{Right:} The same setup but with $\omega_{1\rm e}=10^{-4}$ and $\omega_{2\rm e}=2\times10^{-4}$. The planets are initially caught into resonance around $2\times10^5(2\pi/n_{1,0})$, but then escape resonance around $5\times10^5(2\pi/n_{1,0})$.
    }
    \label{fig:cm}
\end{figure}

\clearpage
\bibliography{bibliography}

\end{document}